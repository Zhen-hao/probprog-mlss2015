\documentclass{article}
\usepackage[utf8]{inputenc}
%\usepackage{listings}
\usepackage{xcolor}
\usepackage{url}
\usepackage{inconsolata}
%\lstset{basicstyle=\ttfamily}
\usepackage{minted}

\title{MLSS 2015 Probabilistic Programming Practical Getting Started Guide}
\author{Frank Wood and Tom Jin}
\date{July 2015}

\begin{document}

\maketitle

\section{Introduction}
The probabilistic programming practical will give you hands-on experience
with probabilistic programming, particularly the Anglican programming language
and system \url{http://www.robots.ox.ac.uk/~fwood/anglican}.

By the end of the practical you should be comfortable programming in Clojure 
and Anglican, and, more importantly, familiar with how to program in probabilistic
programming languages in general.

The exercises and short-lecture introductions to the exercises will be constructed
so as to help you understand how probabilistic programming works in general.  And while the 
amount of time we will have together will be insufficient for you to either implement
a probabilistic programming system or implement a custom inference algorithm in
an existing system, you should be left with a very clear impression of a) how
one would go about implementing such a probabilistic programming language  and 
b) where you would go and what it would take to start
extending, for instance, Anglican with both new probabilistic primitives and 
new inference algorithms.

{\em Prior} to arriving at the practical, please be sure to do the following 
pre-practical preparation (note that while this document is long, the required
preparation work should take, realistically, no more than a few minutes to complete):

\section{Required Pre-Practical Preparation}

Anglican is a language that compiles to Clojure that compiles to the JVM.  For this reason
you need the Java and Clojure ecosystems installed on either your own personal laptop or on a 
machine into which you can ssh, and, in the latter case, to which you can open socket (http) connections.  

\subsection{Java Prerequisites}

Clojure depends on having a JVM installed of version $\geq 1.5$ (the most recent available version is fine).  Windows and Mac OS X users can download Java installers from \url{https://www.java.com/en/download/manual.jsp}. Linux users who do not already have Java installed can install from their package managers:

\begin{minted}{bash}
# Debian/Ubuntu
sudo apt-get install default-jre

# Fedora
sudo yum install java-1.7.0-openjdk
\end{minted}

\subsection{Install Leiningen}

Leiningen is a self-installing automated Clojure project management system.  
You must install Leiningen from \url{http://leiningen.org/}.  ``lein'' 
(short for Leiningen) is a self installing script as well as the primary means
of invoking both Anglican and Clojure read eval print loops (REPL).  Fortunately
``lein'' is trivial to install in *nix environments (see below).  Windows
users have it just as easy but should refer to the website.

\begin{minted}{bash}
# Download lien to ~/bin
mkdir ~/bin
cd ~/bin
wget http://git.io/XyijMQ

# Make executable
chmod a+x ~/bin/lein

# Add ~/bin to path
echo  'export PATH="$HOME/bin:$PATH"' >> ~/.bashrc 

# Run lein
lein
\end{minted}

\subsection{Install Git}

Git is a powerful version control system.  Anglican itself is open source, versioned in Git,
and available at \url{https://bitbucket.org/probprog/anglican}.  The practical will make use
of the Anglican user repository \url{https://bitbucket.org/probprog/anglican-user} which 
provides a bare-bones Leiningen Anglican project which will allow you to get up and running
very quickly.

A short tutorial on Git is available from \url{https://try.github.io}.

\subsubsection{Windows}

An installer is available for download from
\url{http://git-scm.com/download/win}.

\subsubsection{OS X}

If you have Xcode installed then git is already installed, otherwise install the Xcode command line tools which includes git. 

\begin{minted}{bash}
xcode-select --install
\end{minted}

\subsubsection{Linux}

Git is available from your distribution's package manager.

\begin{minted}{bash}
# Debian/Ubuntu
sudo apt-get install git

# Fedora
sudo yum install git
\end{minted}

\subsubsection{Post installation (All platforms)}

Configure git with your name and email address to sign your commits with.
\begin{minted}{bash}
git config --global user.name "Your Name"
git config --global user.email "youremail@domain.com"
\end{minted}


\subsection{Fork and Clone the anglican-user Git Repository}
Here you have an option: if you already have a Bibucket account or don't mind creating one, you may optionally
use a Bitbucket account to fork the anglican-user repository 
and, subsequently clone your fork on your local computer.  

Alternatively you may simply clone, via https, the public anglican-user repo

\begin{minted}{bash}
git clone https://bitbucket.org/probprog/anglican-user.git
\end{minted}

\subsection{You're Done!}

If you have managed to do all this successfully then you will should have a 
Leiningen project called anglican-user sitting locally on your machine.  With it 
you should be able to start a web-based Anglican REPL:
\footnote{Users installing on a server will instead run \texttt{lein gorilla :ip 0.0.0.0} .}

\begin{minted}{bash}
cd anglican-user
lein gorilla
\end{minted}
which will open a browser window which points to a locally hosted url.  If it doesn't 
you should be able to connect to the address printed in the terminal window.  Using the menu on the top right you should be able to open and interactively run \texttt{template.clj}. 

You should be able to start a Clojure command line REPL:

\begin{minted}{bash}
lein repl
\end{minted}

\noindent and within the REPL you should be able to execute the following commands

\begin{minted}{clj}
anglican.core=> (ns mine (:use [anglican core runtime emit]))
mine=> (sample (normal 1 1))
\end{minted}
\noindent which should emit a sample from a $\mathcal{N}(1,1)$ distribution.

\section{Optional Pre-Practical Preparation}

\subsection{IDE Installation}

We will be developing Anglican applications in both the web-based REPL
and standard software development tools.  Clojure
is compatible with a number of IDE's and software development toolchains 
including, for instance, Emacs and Eclipse.  We will use LightTable
to demonstration IDE-based development.  You may use any tool chain you
wish, but LightTable is elegantly compatible with Clojure, if spartan
and initially slightly frustrating.

\url{http://lighttable.com/}

\subsection{Clojure Programming}

Ideally you would be a Clojure programmer, or, at least, a functional 
programming expert in advance of the practical.  Familiarizing yourself
with Clojure can only help the overall experience.  The Clojure main website
\url{http://clojure.org/} has links to a large number of language 
learning resources, in particular \url{http://clojure-doc.org/articles/tutorials/introduction.html}.

\subsection{Probabilistic Programming Reading}

There are a number of probabilistic programming papers and online resources which you 
may wish to read in advance of the practical.  For the purposes of this practical we
specifically recommend reading an introduction to probabilistic programming and the Anglican language \cite{Wood-AISTATS-2014-arXiv-syntax-update-2015} (this is an arXiv update to \cite{Wood-AISTATS-2014} that includes source code in an updated and, as of now, current syntax), a paper on how to implement inference in a probabilistic programming language, in this case probabilistic-C \cite{Paige-ICML-2014}, and a paper about how about higher order probabilistic programming languages can be used to code expressive, particularly Bayesian nonparametric, models \cite{goodman2008church}.

Also there are a number of online resources for Anglican in particular and
probabilistic programming in general.  

\subsubsection{Online Resources}
\begin{itemize}
\item \url{http://www.robots.ox.ac.uk/~fwood/anglican/}
\item \url{http://dippl.org/}
\item \url{http://probabilistic-programming.org/wiki/Home}
\end{itemize}

\bibliographystyle{plain}  
\bibliography{refs}

\end{document}